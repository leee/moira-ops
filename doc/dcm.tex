\documentstyle[fullpage]{article}
\begin{document}
\begin{center}
{\Large The DCM}
\end{center}

The Data Control Manager, or DCM, is the part of Moira that keeps all
of the system servers updated with current data from the Moira
database. It is run by the cron daemon at some time when the database
is not busy. Its operation may be controlled external to Moira by the
presence of {\it /etc/nodcm}, or through Moira by the value of {\bf
dcm\_enable} in the database.

\section{Options}

In normal operation, the DCM will be invoked by {\it startdcm}, which
redirects all logging to {\it /moira/dcm.log}. When invoked by {\it
startdcm}, the DCM checks all services.

Alternately, services can be specified on the command line, in which
case, only those services will be checked.

\section{Operation}

If started with no services specified on the command line, the DCM
first checks for the existence of {\it /etc/nodcm}; if this file
exists, it exits. Next it connects to the database and retrieves the
value of {\bf dcm\_enable} from the values relation of the database; if
this value is zero, it will also exit.

It then scans each of the services which:

\begin{itemize}
\item are {\it enable\/}d
\item do not have {\it hard\_error\/}s
\item the {\it update\_interval\/} is non-zero
\item {\it /moirabin/{\bf service}.gen\/} exists
\end{itemize}

It first compares {\it dfcheck\/} and the {\it update\_interval\/}
against the current time. If it is time for an update, it will run
{\it /moira/bin/{\bf service}.gen}, instructing it to produce {\it
/moira/dcm/{\bf service}.out}. If this finishes without error, {\it
dfgen} and {\it dfcheck} are updated to the current time. If it exits
indicating that nothing has changed, only {\it dfcheck} is updated to
the current time. If there is a soft error (an expected error that
might go away if we try again) then the {\it error\_message} is
updated to reflect this, and nothing else is changed. If there is a
hard error, {\it hard\_error} and {\it error\_message} are set, and a
Zephyr message is sent to class {\bf MOIRA} instance {\bf DCM}.

For each of the services which pass the 4 checks above, regardless of
the result of attempting to build data files, or even if it was not
time to build data files, the hosts will be scanned. The DCM forks off
a separate process to deal with scanning the hosts and propagating the
data to them. (The original DCM process continues scanning the
services.) The hosts are scanned to find those which:

\begin{itemize}
\item are {\it enable\/}d

\item do not have {\it hosterror\/}s

\item have not been successfully updated since the data files were generated
\end{itemize}

For each of these hosts, an update is attempted. First, the data files
are sent to all of the updatable hosts. If it fails to update any
host, it sets the {\it hosterrmsg\/} and {\it ltt\/} (last time tried)
for that host, plus {\it hosterror\/} if a hard error occurred. If the
service is of {\it type\/} {\bf UNIQUE} or {\bf DISTRIB}uted, the DCM
will then continue trying to send the data files to the remaining
hosts. If it is of {\it type\/} {\bf REPLICAT}ed, it will stop there.

After sucessfully sending the data files to all of the hosts (or to
some subset for a {\bf UNIQUE} or {\bf DISTRIB}uted service), the DCM
will attempt to send and run the service {\it script\/} on each host
that received the new data. If it succeeds, it will update the {\it
lts\/} (last time successful) flag for the host, and proceed to the
next. If it fails, it will update {\it hosterrmsg\/} (and, for a hard
failure, it will set {\it hosterror\/} and send a Zephyr warning).

If an error occurs at any point when updating hosts for a {\bf
REPLICAT}ed service, the {\it hosterrmsg\/} and {\it hosterror\/}
values are copied to the service's {\it errmsg\/} and {\it
harderror}, so that the service will be disabled until the error
condition is fixed.


\section{Files}

The source is primarily in {\it dcm/dcm.pc}. The routines to perform a
server update are in {\it update/client.c} and associated files in
that directory.


\section{Data File Generators}

There must be one of these for each service that is supported. Each
generator is an executable program stored in {\it /moira/bin/{\bf
service}.gen}, where {\bf service} is the service name in lower case.
For {\bf UNIQUE} and {\bf REPLICAT}ed services, the DCM will invoke
the generator with one argument, the name of the output file it should
produce. Most of them will write to standard output if no output file
is specified, although this is not required. For {\bf DISTRIB}uted
services, the argument is ignored: they always output files to the
directory {\it /moira/dcm/{\bf service}\/}: one file for each server
to be updated.

The generator exit status should be a value from {\em
$\langle$mr\_et.h$\rangle$}. It must be a value from this set, because
only the least significant 8 bits of the return code are available to
the parent process, so the com\_err table identifier is lost. In
particular, if nothing has changed and the files do not need to be
re-generated, MR\_NO\_CHANGE is returned.

Each generator is different, although there are some conventions.
Most of them check the mod-time on the target file, and compare this
with the mod-times in the tblstats relation in the database for each
table that affects that file.  This determines if the data files need
to be regenerated.  Some of the generators produce a tar file
containing several other files.  These have their working directory
hard-coded in, usually {\it /moira/dcm/{\bf service}}.  Each file that is
produced is written to a different name (usually the desired name with
a twiddle appended), then a rename is performed if the file is
produced successfully.  In this way, if part of the generation fails,
whatever is left should be a set of complete files.

The source to the generators lives in the {\it gen\/} directory.

\end{document}
